\section{Predittori lineari}
Un predittore lineare su $\X=\RN^d$ è una funzione $h:\RN^d\rightarrow\RN$
tale che $h(x)=f(w^\top x)$ con $w\in\RN^d$; $f:\RN\rightarrow\RN$ è
detta funzione di attivazione. 

Nelle attività di regressione lineare, $f$ è la funzione identità, e quindi
$h(x)=w^\top x$.

In quelle di classificazione lineare invece, $h(x)=\sgn(w^\top x-c)$ con
$c\in\RN$, dove:
$$ \sgn(z) = \begin{cases}1&z>0\\0&\text{altrimenti}\end{cases} $$

\subsection{Iperpiani}
Un iperpiano può essere descritto da un'equazione lineare della seguente
forma:
$$ w_1x_1+w_2x_2+\dots+w_nx_n=c $$
$$ \downarrow $$
$$ \underbrace{\begin{bmatrix}w_1\\w_2\\\vdots\\w_n\end{bmatrix}^\top}
      _{\displaystyle w}\cdot
   \underbrace{\begin{bmatrix}x_1\\x_2\\\vdots\\x_n\end{bmatrix}}
      _{\displaystyle x}=c
$$
$$ \downarrow $$
$$ w^\top x = c $$
Si può quindi dire che un iperpiano con coefficienti $(w,c)$ è definito da
$\{x\in\RN^d:w^\top x=c\}$ dove:
\begin{equation}
   w^\top x = \|w\|\cdot\|x\|\cos\theta\label{eq:hyperplane_eq}
\end{equation}
e $\theta$ è l'angolo tra $w$ e $x$.

I sempispazi $H^+$ e $H^-$ definiti dall'iperpiano $\{x\in\RN^d:w^\top x=c\}$
sono:
$$ H^+=\{x\in\RN^d:w^\top x>c\} $$
$$ H^-=\{x\in\RN^d:w^\top x\leq c\} $$

Geometricamente, un classificatore lineare è definito da:
$$ h(x) = \begin{cases}
+1 & x\in H^+\\
-1 & x\in H^-\\
\end{cases} $$

\subsection{Classificatori lineari}
Si ricordi che un classificatore lineare è un predittore $h$ tale che
$h(x)=\sgn(w^\top x)$. Applicando (\ref{eq:hyperplane_eq}) si ottiene che:
$$ \begin{aligned}
   h(x)&=\sgn(w^\top x)\\
   &=\sgn(\|w\|\cdot\|x\|\cos\theta)\\
   &=\sgn(\cos\theta)\\
\end{aligned} $$
Questo ci mostra che il valore di $\|w\|$ è irrilevante e si dovrebbe
prendere $\|w\|=1$. Inoltre la \textit{zero-one loss} $\I\{h(x_t)\neq y_t\}$
può essere riscritta come $\I\{y_tw^\top x_t\leq 0\}$.

Sia $\Hpred_d$ la famiglia di classificatori lineari $h(x)=\sgn(w^\top x)$
con $w \in \RN^d:\|w\|=1$. Si consideri l'algoritmo ERM per la 
\textit{zero-one loss}:

\begin{align}
h_S &= \argmin_{h\in\Hpred_d}\frac{1}{m}\sum_{t=1}^{m}\I\{h(x_t)\neq y_t\}
   \notag\\
&= \argmin_{w\in\RN:\|w\|=1}\frac{1}{m}\sum_{t=1}^{m}\I\{y_tw^\top x_t\leq 0\}
\label{eq:lin_pred_01_ERM}
\end{align}

\textbf{Sfortunatamente, è improbabile esista un implementazione efficiente
di ERM per classificatori lineari con la \textit{zero-one loss}}. Infatti,
il problema di decisione associato alla ricerca di $h_S$ è NP-completo.

\subsubsection{Separabilità lineare}

Il problema ERM (\ref{eq:lin_pred_01_ERM}) diventa più semplice quando il 
\textit{training set} è linearmente separabile, ovvero quando esiste un 
classificatore lineare con \textit{training error} nullo, e quindi si può 
dividere linearmente tutti i \textit{data point} in etichette differenti.

Si noti che il problema (\ref{eq:lin_pred_01_ERM}) può essere rappresentato 
da un sistema di disequazioni lineari:
$$ y_tw^\top x_t >0 \qquad t=1,\dots,m $$
$$ \downdownarrows $$
$$ \begin{cases}
y_1 w^\top x_1 >0 \\
\qquad\vdots \\
y_m w^\top x_m >0 \\
\end{cases} $$
\textbf{Se il \textit{training set} è linearmente separabile allora il
sistema ha almeno una soluzione, che può essere trovata in tempo polinomiale
utilizzando un algoritmo per la programmazione lineare}.

Per risolvere il sistema di disequazioni si può usare l'algoritmo del
simplesso. Tuttavia esiste un algoritmo ancora più semplice: 
\textbf{l'algoritmo \textit{Perceptron}}.

\subsubsection{\textit{Perceptron}}
Il \textit{Perceptron} è in grado di risolvere il problema 
(\ref{eq:lin_pred_01_ERM}) in presenza di un \textit{training set} 
linearmente separabile. Per trovare l'iperpiano separatore scorre gli esempi
del \textit{training set} uno dopo l'altro. Partendo da un classificatore
lineare iniziale, lo testa su ogni esempio e, in caso di errore, corregge
l'iperpiano associato.

\begin{algorithm}[h]
   \DontPrintSemicolon
   \KwData{\textit{Training set} $(x_1,y_1),\dots,(x_m,y_m)$}
   $w=(0,\dots,0)$\;
   \While{$true$}{
      \For(\hspace{2.7em}\tcp*[h]{epoca}){$t=1,\dots,m$}{
         \uIf{$y_t w^\top x_t \leq 0$}{
            $\displaystyle w \leftarrow w+y_tx_y\qquad$
            \tcp*[h]{aggiornamento} \;
         }
      }
      \lIf{nessun aggiornamento nell'ultima epoca}{\textbf{break}}
   }
   \KwOut{$w$}
   \caption{\textit{Perceptron} (per i casi linearmente separabili)}
\end{algorithm}

Geometricamente, ogni aggiornamento sposta l'iperpiano $w$ verso $x_t$ se
$y_t=1$; se invece $y_t=-1$, $w$ viene allontanato da $x_t$.

Si può dimostrare che il \textit{Perceptron} termina sempre in presenza
di un \textit{training set} linearmente separabile.

\begin{theorem}[Convergenza di \textit{Perceptron}]
   Sia $(x_1,y_1),\dots,(x_m,y_m)$ un \textit{training set} linearmente
   separabile. Allora il Perceptron terminerà sempre dopo un numero di
   aggiornamenti non più grande di:
   $$ \left(\min_{\displaystyle u:\gamma(u)\geq1} \|u\|^2 \right)\cdot
      \left(\max_{\displaystyle t=1,\dots,m}\|x_t\|^2\right) $$

   dove $\gamma(u)$ è il punto più vicino all'iperpiano $u$:
   $$ \gamma(u) = \min_{t=1,\dots,m} y_t u^\top x_t $$
\end{theorem}

Si noti che questo teorema non implica che il \textit{Perceptron} termini
in tempo polinomiale.

\subsection{Regressione lineare}
Nella regressione lineare i predittori sono funzioni lineari
$h:\RN^d\rightarrow\RN$ parametrizzate da un vettore $w\in\RN^d$ di
coefficienti reali, ovvero:
$$ h(x) = w^\top x $$
I coefficienti del predittore dell'algoritmo ERM per la \textit{square loss}
la regressione lineare sono:
$$ \begin{aligned}
   w_S &= \argmin_{w\in\RN^d}\sum_{t=1}^{m} (w^\top x_t - y_t)^2\\
   &= \argmin_{w\in\RN^d} \|Sw-y\|^2
   \end{aligned}
$$

\subsubsection{\textit{Ridge Regression}}
Un predittore più stabile di ERM è ottenuto introducendo un regolarizzatore
che aumenta l'errore di approssimazione e riduce la varianza con un effetto
positivo sul rischio. La forma regolarizzata di $w_S$ è:
$$\begin{aligned}
   w_{S,\alpha} &= \argmin_{w\in\RN^d}\|S_w-y\|^2+\alpha\|w\|^2\\
   &= (\alpha I+S^\top S)^{-1}S^\top y
\end{aligned}$$
con $\alpha>0$