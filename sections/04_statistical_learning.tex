\section{\textit{Statistical Learning}}

Per poter analizzare un \textit{learning algorithm} c'è bisogno di definire un modello
matematico di come gli esempi $(x,y)$ siano generati. Nel contesto della \textit{statistical
learning} si assumerà che ogni esempio sia ottenuto attraverso un'estrazione indipendente
da una distribuzione di probabilità fissata su $\X \times \Y$. Si scriverà $(X,Y)$ per
sottolineare come \textbf{le due componenti di un esempio siano due variabili aleatorie}.

Assumere che ogni esempio $(x,y)$ sia la realizzazione di un'estrazione casuale 
\textbf{indipendente} da un'unica distribuzione $\D$, implica che ogni \textit{dataset} 
(come \textit{test} e \textit{training set}) sia un campione statistico. L'indipendenza dei
dati è in realtà violata in alcuni domini pratici. Nonostante ciò, l'assunzione di indipendenza
nei dati è estremamente utile dal punto di vista della tracciabilità analitica del problema
e funziona sorpendentemente bene nella pratica.

\subsection{Definizioni}
Nel contesto della \textit{statistical learning} un problema è specificato da una coppia
$(\D,\ell)$, dove $\D$ è la distribuzione e $\ell$ la \textit{loss function}.

\subsubsection{Rischio statistico \texorpdfstring{$\ell_{\D}$}{lD}}
Le prestazioni di un predittore $h:\X \rightarrow \Y$ rispetto a $(\D,\ell)$ è valutata dal
\textbf{rischio statistico}:
$$ \ell_{\D}(h) = \E[\ell(Y,h(X))] $$
che indica il valore atteso della \textit{loss function} su un esempio casuale $(X,Y)$
estratto da $\D$.

\subsubsection{Predittore ottimo di Bayes \texorpdfstring{$f^*$}{f*}}
Data $\D$, il miglior predittore possibile $f^*:\X \rightarrow \Y$ è detto 
\textbf{predittore ottimo di Bayes}:
$$ f^*(x) = \argmin_{\hat{y} \in \Y} \E[\ell(Y,\hat{y})\ | \ X=x] $$

\subsubsection{Rischio condizionato}
L'argomento di $\argmin$ di $f^*$, ovvero $\E[\ell(Y,\hat{y})\ | \ X=x]$, è detto
\textbf{rischio condizionato}. Il \textit{Bayes optimal predictor} quindi, è la predizione
che minimizza il rischio condizionato. Un altro modo per scrivere il rischio condizionato
di un predittore $f$ è:
$$ \E[\ell(Y,\hat{y})\ | \ X=x] = \E[\ell(Y,f(X))\ | \ X=x] $$

\subsubsection{Rischio di Bayes \texorpdfstring{$\ell_{\D}(f^*)$}{lDF*}}
Essendo $f^*$ il miglior predittore possibile per $\D$ è ragionevole pensare che esso
abbia il rischio statistico migliore, ed è così; si ha infatti che il rischio 
$\ell_{\D}(f^*)$, detto \textbf{rischio di Bayes}, è il minore tra tutti i predittori:
$$ \forall h \in \F \quad \ell_{\D}(f^*) \leq \ell_{\D}(h) $$
Tipicamente il rischio di Bayes è maggiore di zero vista la casualità delle etichette.

\subsection{\texorpdfstring{$f^*$}{f*} e \texorpdfstring{$\ell_\D$}{lD*} nelle varie \textit{loss function}}
Si valuteranno ora i predittori ottimi di Bayes per le varie \textit{loss function}.
\subsubsection{\textit{Quadratic loss}}

$$ \ell(y,\hat{y}) = (y-\hat{y})^2 \qquad \text{con } \Y = \RN $$
\begin{align}
    f^*(x) &= \argmin_{\hat{y}\in\RN} \E[(Y-\hat{y})^2 \ | \ X=x]  \notag\\
           &= \argmin_{\hat{y}\in\RN} \E[Y^2+\hat{y}^2-2\hat{y}Y\ | \ X=x]\notag\\[.6em]
        \multispan2{Per le varie proprietà del valore atteso si ha:\hfil} \notag\\[.6em]
           &= \argmin_{\hat{y}\in\RN}
            \left({\E[Y^2\ | \ X=x]}+\E[\hat{y}^2\ | \ X=x]-\E[2\hat{y}Y\ | \ X=x]\right)
            \notag\\[.6em]
           &= \argmin_{\hat{y}\in\RN}
           \left({\color{red}\E[Y^2\ | \ X=x]}+\E[\hat{y}^2\ | \ X=x]-2\hat{y}\E[Y\ | \ X=x]\right)
           \notag\\[.6em]
        \multispan2{Siccome $\argmin$ varia su $\hat{y}$, tutti {\color{red}i fattori che non ne dipendono}\hfil
        } \notag\\
        \multispan2{non incidono sul risultato; possono quindi essere tolti:\hfil
        } \notag\\[.6em]
        &= \argmin_{\hat{y}\in\RN}
           \left(\E[{\color{blue}\hat{y}}^2\ | \ X=x]-2\hat{y}\E[Y\ | \ X=x]\right)
           \notag\\[.6em]
        \multispan2{Non essendo ${\color{blue}\hat{y}}$ una variabile aleatoria:\hfil
           } \notag\\[.6em]
        &= \argmin_{\hat{y}\in\RN}
            \left({\color{blue}\hat{y}^2}-2\hat{y}\E[Y\ | \ X=x]\right) \notag
\end{align}

L'argomento di $\argmin$ è una funzione del tipo:
\begin{equation} F(\hat{y}) = \hat{y}^2 -2\hat{y}q \tag*{$q=\E[Y\ | \ X=x]$}\end{equation}
Obiettivo di argmin è trovare il valore di $\hat{y}$ che minimizza $F$. Facendo un semplice 
studio di funzione si può trovare che $F$ è minimizzata quando:
$$
\begin{aligned}
    \multispan2{$F'(\hat{y}) = 2\hat{y}-2q$} \\
    \multispan2{Cerco il minimo:\hfill}\\
    F'(\hat{y}) &= 0\\
    2\hat{y}-2q &= 0\\
    \hat{y}     &=q\\
\end{aligned} \qquad \Rightarrow \qquad \hat{y}=\E[Y\ | \ X=x]
$$

Si può quindi dire che:
$$ f^*(x) = \E[Y\ | \ X=x]  = \E[Y\ | \ X]$$

Sostituendo il risultato appena mostrato nella formula del rischio condizionato si ha:
$$ \begin{aligned}
    \ & \E[\ell(Y,\hat{y})\ | \ X=x] \\ 
    = \ &\E[(Y-f^*(X))^2\ | \ X=x] \\
    = \ &\E[(Y-\E[Y \ | \ X ])^2\ | \ X=x] \\
    = \ & \var[Y \ | \ X]
\end{aligned} $$

Il rischio di Bayes sarà quindi:
$$ \ell_\D(f^*) = \E[\var[Y \ | \ X]] \neq \var[Y] $$

Da notare che il valore atteso della varianza dato $X$ sia diverso dalla varianza; per
la legge della varianza totale si ha infatti che:
$$ \var[Y]-\E[\var[Y \ | \ X]] = \var[\E[Y \ | \ X]] $$



\subsubsection{\textit{Zero-one loss}}
