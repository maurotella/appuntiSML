\section{Tree Predictors}

Come già visto, mentre alcuni tipi di dato hanno una naturale rappresentazione
vettoriale $x \in \RN^d$, altri non ce l'hanno. Un esempio possono essere dei
\textit{record} medici, dove i dati contengono i seguenti campi:
$$ \begin{aligned}
    &\texttt{età} \in \{12,\dots,90\} \\
    &\texttt{fumatore} \in \{\text{sì},\text{no},\text{ex}\} \\
    &\texttt{peso} \in [10,200] \\
    &\texttt{sesso} \in \{\text{M},\text{F}\} \\
    &\texttt{terapia} \in \{\text{antibiotici},\text{cortisone},\text{nessuna}\} \\
\end{aligned} $$

Anche convertendo questi tipi di dato in dati numerici, gli algoritmi basati sulla
distanza euclidea, come il \kNN, potrebbero non andare molto bene.

\textbf{Per poter applicare la \textit{data inference} su dati le cui 
\textit{feature} variano in insiemi eterogenei $\X_1,\dots,\X_d$, verrà introdotta 
una nuova famiglia di predittori: i \textit{tree predictors}}.

Un \textit{tree predictor} è un albero ordinato e radicato dove ogni nodo può
essere una \textbf{foglia} o un \textbf{nodo interno}. In figura \ref{fig:tree_class}
viene mostrato un esempio di \textit{tree predictor} binario le cui \textit{feature}
sono:

$$ \begin{aligned}
    &\texttt{previsione} \in \{\text{sole},\text{nuvole},\text{pioggia}\} \\
    &\texttt{umidità} \in [0,100] \\
    &\texttt{vento} \in \{\text{sì},\text{no}\}
\end{aligned} $$
 
\begin{figure}[h]
    \centering
    \usetikzlibrary {shapes.geometric}
\begin{tikzpicture}[sibling distance=18mm, level distance=20mm]
    
    \node [ellipse,draw] {\texttt{previsione}}
        child {node[ellipse,draw] {\texttt{umidità}}
            child {node[rectangle,draw] {$+1$} edge from parent node[above,rotate=63] {$\leq 70\%$}}
            child {node[rectangle,draw] {$-1$} edge from parent node[above,rotate=-65] {$> 70\%$}}
            edge from parent node[above,rotate=46] {sole}
        }
        child {node[rectangle,draw] {$+1$} edge from parent node[above,rotate=90] {nuvole}}
        child {node[ellipse,draw] {\texttt{vento}} 
            child {node[rectangle,draw] {$-1$} edge from parent node[above,rotate=63] {sì}}
            child {node[rectangle,draw] {$+1$} edge from parent node[above,rotate=-65] {no}}
            edge from parent node[above,yshift=-.3em,xshift=.2em,rotate=-51] {pioggia}
        };

\end{tikzpicture}
    \caption{Esempio classico di \textit{tree classifier} binario.\label{fig:tree_class}}
\end{figure}

