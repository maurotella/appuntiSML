\section{Tree Predictors}

\subsection{Definizione}
Come già visto, mentre alcuni tipi di dato hanno una naturale rappresentazione
vettoriale $x \in \RN^d$, altri non ce l'hanno. Un esempio possono essere dei
\textit{record} medici, dove i dati contengono i seguenti campi:
$$ \begin{aligned}
    &\texttt{età} \in \{12,\dots,90\} \\
    &\texttt{fumatore} \in \{\text{sì},\text{no},\text{ex}\} \\
    &\texttt{peso} \in [10,200] \\
    &\texttt{sesso} \in \{\text{M},\text{F}\} \\
    &\texttt{terapia} \in \{\text{antibiotici},\text{cortisone},\text{nessuna}\} \\
\end{aligned} $$

Anche convertendo questi tipi di dato in dati numerici, gli algoritmi basati sulla
distanza euclidea, come il \kNN, potrebbero non andare molto bene.

\textbf{Per poter applicare la \textit{data inference} su dati le cui 
\textit{feature} variano in insiemi eterogenei $\X_1,\dots,\X_d$, verrà introdotta 
una nuova famiglia di predittori: i \textit{tree predictors}}.

Un \textit{tree predictor} è un albero ordinato e radicato dove ogni nodo può
essere una \textbf{foglia} o un \textbf{nodo interno}. È importante sottolineare che
in un albero ordinato i figli di ogni nodo sono anch'essi ordinati e quindi numerabili
consecutivamente.
In figura \ref{fig:tree_class}
viene mostrato un esempio di \textit{tree predictor} binario le cui \textit{feature}
sono:
$$ \begin{aligned}
    &\texttt{previsione} \in \{\text{sole},\text{nuvole},\text{pioggia}\} \\
    &\texttt{umidità} \in [0,100] \\
    &\texttt{vento} \in \{\text{sì},\text{no}\}
\end{aligned} $$
 
\begin{figure}[h]
    \centering
    \usetikzlibrary {shapes.geometric}
\begin{tikzpicture}[sibling distance=18mm, level distance=20mm]
    
    \node [ellipse,draw] {\texttt{previsione}}
        child {node[ellipse,draw] {\texttt{umidità}}
            child {node[rectangle,draw] {$+1$} edge from parent node[above,rotate=63] {$\leq 70\%$}}
            child {node[rectangle,draw] {$-1$} edge from parent node[above,rotate=-65] {$> 70\%$}}
            edge from parent node[above,rotate=46] {sole}
        }
        child {node[rectangle,draw] {$+1$} edge from parent node[above,rotate=90] {nuvole}}
        child {node[ellipse,draw] {\texttt{vento}} 
            child {node[rectangle,draw] {$-1$} edge from parent node[above,rotate=63] {sì}}
            child {node[rectangle,draw] {$+1$} edge from parent node[above,rotate=-65] {no}}
            edge from parent node[above,yshift=-.3em,xshift=.2em,rotate=-51] {pioggia}
        };

\end{tikzpicture}
    \caption{Esempio classico di \textit{tree classifier} binario.\label{fig:tree_class}}
\end{figure}

Sia $ \X = \X_1,\dots,\X_d $, dove ogni $\X_i$ rappresenta il dominio dell'$i$-esimo
attributo (o \textit{feature}) $x_i$. \textbf{Il \textit{tree predictor} $h_T:\X 
\rightarrow \Y$ è un predittore definito da un albero $T$ i cui nodi interni 
corrispondono a dei test e le cui foglie corrispondono a delle etichette $y \in \Y$}.

Un test su un attributo $i$ su un nodo interno con $k$ figli è una funzione 
$f:\X \rightarrow \{1,\dots,k\}$. $f$ mappa ogni elemento di $\X_i$ a un nodo figlio. 
Due esempi possono essere i seguenti:

\begin{minipage}{.45\textwidth}
    $$\X_i = \{a,b,c,d\} \qquad\quad k=3$$
    $$ f(x_i) = 
        \begin{cases}
            \ 1 & x_i = c\\
            \ 2 & x_i = d\\
            \ 3 & x_i \in \{a,b\}\\
        \end{cases}
    $$
\end{minipage}
\begin{minipage}{.45\textwidth}
    $$\X_i = [0,100] \qquad\quad k=2$$
    $$ f(x_i) = 
        \begin{cases}
            \ 1 & x_i \in [0,70]\\
            \ 2 & x_i \in (70,100]\\
        \end{cases}
    $$
\end{minipage}

L'esempio di destra è riferito all'attributo \texttt{umidità} di figura
\ref{fig:tree_class}.

La predizione $h_T(x)$ è calcolata come segue:
\begin{enumerate}
    \item $v \leftarrow r \qquad$ ($r$ è la radice di $T$)
    \item se $v$ è una foglia $\ell$, si restituisce l'etichetta $y \in \Y$
    associata a $\ell$;
    \item altrimenti, sia $f:\X_i \rightarrow \{1,\dots,k\}$ il test associato a
    $v$, \textbf{assegna $v\leftarrow v_j$} dove $j=f(x_i)$ e $v_j$ indica il
    $j$-esimo figlio di $v$;
    \item vai al passaggio 2.
\end{enumerate}

Se $h_T(x)$ restituisce la foglia $\ell$, si dirà che l'esempio $x$ è indirizzato a
$\ell$.

\subsection{Costruzione di un \textit{tree predictor}}
Dato un \textit{training set} $S$, si vedrà ora come costruire un \textit{tree predictor}.
Per semplicità, si guarderà solo ad una classificazione binaria $\Y=\{-1,1\}$ e verranno
usati solo alberi binari completi, cioè alberi dove ogni nodo interno ha due figli.
