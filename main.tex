\documentclass[a4paper]{article}

\usepackage[top=2.5cm]{geometry}
\usepackage{lmodern}
\usepackage{parskip}
\usepackage{amsmath}
\usepackage{amsfonts}

\begin{document}

\title{Statistical methods for machine learning}
\author{Mauro Tellaroli}
\date{}
\maketitle

\section{Introduzione}

La \textit{data inference} è lo studio dei metodi che utilizzano i dati per predirre il futuro. 
Il \textit{Machine Learning} è uno strumento potente che può essere usato per risolvere una 
grossa parte dei problemi di \textit{data inference}, inclusi i seguenti:
\begin{itemize}
    \item \textbf{Clustering}: raggruppare i \textit{data points} in base alle loro similarità;
    \item \textbf{Prediction}: assegnare delle etichette (\textit{label}) ai \textit{data points};
    \item \textbf{Generation}: generare nuovi \textit{data points};
    \item    \textbf{Control}: eseguire una sequenza di azioni in un ambiente con l'obiettivo di
                               massimizzare una nozione di utilità.
\end{itemize}

Con \textit{data point} si intende una serie di informazioni legate ad un unico elemento;
un'analogia può essere un \textit{record} in un database.

Gli algoritmi che risolvono una \textit{learning task} in base a dei dati già semanticamente
etichettati lavorano in modalità \textbf{\textit{supervised learning}}. A etichettare i dati
saranno delle persone o la natura. Un esempio dell'ultimo caso sono le previsioni del meteo. 
D'altra parte, gli algoritmi che utilizzano i dati senza la presenza di etichette lavorano in
modalità \textbf{\textit{unsupervised learning}}.

In questo corso ci si focalizzerà sul \textit{supervised learning} e la progettazione di 
sistemi di \textit{machine learning} il cui obiettivo è apprendere dei 
\textbf{\textit{predictors}}, ovvero funzioni che mappano i \textit{data points} alla loro
etichetta.

\subsection*{Label set $\mathcal{Y}$}
Verrà usata $\mathcal{Y}$ per indicare il label set, ovvero l'insieme di tutte le possibili
etichette di un \textit{data point}. Le etichette potranno essere di due tipi differenti:
\begin{enumerate}
    \item \textbf{Categoriche} ($\mathcal{Y} = \{ \text{sport},\text{politica},\text{economia}\}$):
        si parlerà di problemi di \textbf{classificazione};
    \item \textbf{Numeriche} ($\mathcal{Y} \subseteq \mathbb{R} $): 
        si parlerà di problemi di \textbf{regressione}.
\end{enumerate}







\end{document}